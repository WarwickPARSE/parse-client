\chapter{Design}
\label{design}
This Chapter outlines the overall design of the PARSE toolkit and the design choices made by the group. Each aspect of the system has required prototyping in order to evaluate the best approach to overall integration of the system and so that the requirements in the specification can be met. A high level system overview is presented along with detailed design discussions of each functional part of the system. These include reference to key algorithms, operations and run-time analysis where appropriate. The design section concludes with a consideration of the interface design of the system.

\section{System Implementation}

The implementation of the system describes each component as it was implemented in the final system. In this section, the broad functionality of each component is described. Further subsections discuss the overall accuracy and results achieved from the components implemented. \\

\subsection{CoreLoader}

The main Coreloader screen contains all functionality associated with the system. There are options present for selecting particular Kinect feeds, calibrating the Kinect's position and viewing the debug console. Recent patients are also displayed persistently for quick access to the scans and results of previous scans on a particular patient. Configuration options are also possible through Coreloader such as setting the working directory for PARSE files and attributing .PARSE and .PCD scan files to patients. Exporting scans to .PCD is also here so that scans produced by the System can be visualised using the Point Cloud Library toolkit\footnote{Point Cloud Library, http://www.pointclouds.org}. \\

\begin{center}
    \includegraphics[scale=0.5]{zscreenshots/corewelcome.png}\\
    \caption{CoreLoader welcome screen with toolkit options}
\end{center} \\

\begin{center}
    \includegraphics[scale=0.6]{zscreenshots/savetopcd.PNG}\\
    \caption{Confirmation of point cloud file exported to .PCD}
\end{center} \\

\begin{center}
    \includegraphics[scale=0.55]{zscreenshots/metaloader.PNG}\\
    \caption{List of patients in the database with associated point cloud or scan information}
\end{center} \\

\begin{center}
    \includegraphics[scale=0.3]{zscreenshots/coreloaderstef.PNG}\\
    \caption{System with scan and associated patient detail displayed}
\end{center} \\


\subsection{ViewLoader}

\begin{center}
    \includegraphics[scale=0.5]{zscreenshots/rgbvisualiser.PNG} \\
    \caption{ViewLoader: Showing the raw RGB feed}
\end{center} \\

Viewloader is capable of displaying raw streams from the Kinect or composite streams that combine multiple views such as depth isolation, colour isolation, or the overlay of a skeleton onto existing feeds. The above image shows the conventional RGB feed from the Kinect. \\

\subsection{ScanLoader}

\begin{center}
    \includegraphics[scale=0.3]{zscreenshots/patientloading.png} \\
    \caption{ScanLoader: The patient data portal (left) and the model viewer loading (right)}
\end{center} \\

\begin{center}
    \includegraphics[scale=0.5]{zscreenshots/scanframework.png} \\
    \caption{ScanLoader: 3D model viewport}
\end{center} \\

ScanLoader visualises the registered point clouds from the scan process and allows the model to be viewed from a number of different perspectives and zoom levels. The point cloud visualisation process is resource intensive as approximately 200,000 points are drawn in 3D space. As a result of this, a persistent loading dialog is displayed when the point cloud is being loaded into scan loader or when refining operations are applied to the point cloud. \\

\subsection{HistoryLoader}

\begin{center}
    \includegraphics[scale=0.4]{zscreenshots/circumdetail.png}
    \caption{HistoryLoader: Circumference history}
\end{center} \\

\begin{center}
    \includegraphics[scale=0.4]{zscreenshots/voldetail.png}
    \caption{HistoryLoader: Volume, height \& plane history}
\end{center} \\

\begin{center}
    \includegraphics[scale=0.75]{zscreenshots/volumevisualisation.png}
    \caption{HistoryLoader: Plane viewer}
\end{center} \\

The HistoryLoader component of the system visualises the volume and circumference calculation results along with visualisations related to the extrcted planes from the point cloud as well as the visualisation for the approximation of the circumference of the limbs. In each component, different metrics can be visualised for volume in terms of the different planes of the body as well as the circumference for each limb that has been partitioned. These results can then be saved or refined during a rescan procedure. \\

\subsection{PatientLoader}

\begin{center}
    \includegraphics[scale=0.5]{zscreenshots/patientdetail.png} \\
    \caption{PatientLoader: Patient information pane}
\end{center} \\

The PatientLoader presents the patient detail recovered from the database when a patient's scans for either volume/limb circumference or markerless recognition purposes is loaded. PatientLoader also provides the ability to apply new markerless measurements and recover old measurements. Patient information such as personal details and patient conditions can also be added here. \\

\subsection{MeasurementLoader}
MeasurementLoader presents the mechanism for adding a new body-relative scan position with a hand-held sensor. Clear instructions are displayed, indicating to the users how to progress.\\

\begin{enumerate}
    \item Wait for two people to be in view
    \item Search for the sensor device
    \item Use the sensor's location, in combination with the skeletal data, to identify the doctor and the patient
    \item When the sensor is held still for 10 seconds, the system automatically runs the capture routine.
\end{enumerate} \\

\begin{center}
    \setlength\fboxsep{0pt}
    \setlength\fboxrule{0.5pt}
    \fbox{\includegraphics[scale=0.25]{zscreenshots/MeasurementLoader.png}}\\
    \caption{MeasurementLoader: Ready to start a scan}
\end{center} \\

\begin{center}
    \setlength\fboxsep{0pt}
    \setlength\fboxrule{0.5pt}
    \fbox{\includegraphics[scale=0.25]{zscreenshots/MeasurementLoader2.png}}\\
    \caption{MeasurementLoader: The large, clear display indicates the next step}
\end{center}

\section{Person Isolation}
\label{design:person isolation}
The group decided to make use of the Kinect's abilty to generate a skeleton frame associated with a person to aid in isolation. This decision meant the group stepped away from the generic computer vision algorithms discussed in Section \ref{person_isolation:specific algorithms}, such as KDE and MOG.\\

Two possible methods of person isolation were designed by the group. The first method made use of the Kinect colour stream whereas the second used the Kinect depth stream. Both methods described operate on a per pixel basis, have a complexity of $O(n)$ where $n$ is the number of pixels in a frame and are theoretically capable of isolation stationary people, which the reserached methods may not be able to do.\\

\subsection{Colour Based Isolation}
\label{design:colour based isolation}
In this algorithm, the pre processing is handled by the Kinect API, converting the raw infra red depth data into a byte array. The foreground mask is calculated using the Kinect API to determine whether a colour pixel is associated with a detected skeleton. If a pixel is associated with a skeleton, the pixel is a foreground pixel. And conversely, an unassociated pixel is a background pixel. There is no data validation in this algorithm.\\

At this stage it was expected that colour based isolation would preform well, but it was unknown if a point cloud could be constructed using this method, as the colour data contains no depth.\\

\subsection{Depth Based Isolation}
\label{design:depth based isolation}

In this algorithm, the pre processing is again handled by the Kinect API, converting the raw infra red depth data into a byte array. This algorithm then make use of the skeleton to determine the approximate depth of the person. Any pixel whose depth value is outside a delta of the skeleton's depth would be considered a background pixel. Cutting off based on depth alone is not enough, as doing so would leave a ring of equidistant points in-line with the person.\\

To eliminate this ring, the positions of left and right most point of the person (i.e. the HandLeft joint and the HandRight joint) would also be used for cut off and anything outside of this range would also be classified as a background pixel and discarded. All other pixels would be considered foreground and again there is no data validation phase.\\ 

This method would leave a square of floor under the person, but at the design stage it was hoped this could be removed at the point cloud level. Whilst a depth based cut off may not be as effective at removing all the miscellaneous non-person data, i.e. the floor, it may be better suited to creating a point cloud than colour based isolation.\\
%complete
\section{Point Clouds}
The term \emph{Point Cloud} describes both a conceptual object and a data structure. The conceptual object is either the model or the data which has been described in Sections \ref{research:registration} and \ref{res:range imaging} respectively. The Point Cloud data structure and class design will now be considered, as its functionality forms an integral part of many of the algorithms that will follow. \\

\subsection{Data Structures}
A number of data structures exist within the point cloud. The primary data structure represents a point cloud and there are various secondary data structures containing meta-information. The point cloud itself consisted of numerous \texttt{PointRGB} data structures. \\

Initially depth data from the Kinect device is structured in a single-dimensional array of x, y, z coordinates as well as some un-needed data. This data structure is by no means easy to directly access and some abstraction was required. Texture data is stored in an entirely different data structure in the form of a buffered Bitmap image. \\

The first solution was to place the data in an $N \times 3$ matrix. This created a relatively intuitive access method although finding a nearest neighbour would take $O(N)$ time which was unacceptable considering the large amount of point data coming from the Kinect device. This would have resulted in some of the algorithms operating in $O(N^2)$ time. \\

The second solution was created to satisfy the needs of the algorithms to be implemented. Point data is stored in a \emph{K-Dimensional} (K-D) tree. In addition to standard operations such as \emph{addition} and \emph{deletion}; the K-Dimensional tree supports \emph{exact matching}, \emph{partial matching}, \emph{range queries} and \emph{nearest neighbour searches} \cite{bentley90}. \\

\subsubsection{PointRGB}
\label{pointrgb}
The point cloud data structure could only store \emph{key, value} pairs. The data that was required, however consisted of individual \emph{red}, \emph{blue}, \emph{green} and \emph{Point} data items. This issue was overcome by encapsulating these data items into a C# \texttt{struct}.\\ % I refuse to acknowledge that data is the plural of data

\subsection{Algorithms}
\paragraph{Converting streams to K-D trees}
As discussed in the previous section, depth data arrives from the Kinect device in quite an unmanageable form. This data then has to be processed so that it can be easily imported into a K-D tree data structure. It was anticipated that the point cloud data structure may be used in different ways and so storage of texture information is optional. \\

The simplest and computationally cheapest way to insert the data into the K-D tree was to convert the texture information into a form that is similar to the input depth data, that is multiple arrays representing each of the primary colours. \\

The depth and texture arrays are then input into a method that converts the Kinect co-ordinate system into real-world coordinates and then wraps the information for each point into a \texttt{PointRGB} (Section \ref{pointrgb}). \\

Finally, the point data is inserted into the K-D tree using the point $x$, $y$ and $z$ coordinates as the key. 

\subsection{Visualisation}
To determine the quality of registration and for aesthetic reasons a visualisation of the point cloud was designed. The visualisation was to model each depth point in the point cloud as a uniform 3-D shape that is pleasing to the eye. 

\subsection{Interim Summary}
The data structures and algorithms required for efficient point cloud processing, while increasing code reusage, are all stored within the Point Cloud data structure. \\ 
\section{Volume Estimation}
\label{design:volume estimation}
This section will focus on the group's algorithm design for volume estimation. Following researched in Section \ref{research:ssp}, SSP and planimetry were the basis for the groups design

\subsection{Planimetry}
\label{design:planimetry}
SSP produces high accuracy approximations to volumes. The group decided to design an algorithm based on the volume rendering approach of SSP. The algorithm would pull successive planes out of the point cloud and apply the Shoelace formula to determine the area of a plane.\\

As any volume would be calculated in ``point cloud space", the volume would not necessarily correspond directly to SI units such as meters. Hence, the volumes returned by the SSP method may need to by multiplied by a constant in order for an accurate volume to be returned. This multiplicative can only be determined through testing.\\

\subsection{Algorithm Running Time}
For a given plane, the Shoelace formulae is applied to the sorted list of points in $O(p')$, where $p'$ is the number of points in the plane and $p' \in O(p)$. Hence, the area of an individual plane can be calculated in $O(p)$.\\ 

There are $O(y)$ number of planes, where $y$ is the number of points in the y axis (the axis corresponding to height) and $O(y) \in O(p)$ for simplicities sake. Hence the volume calculation runs with complexity $O(p^2)$. As circumference and area can also be calculated using variations of the Shoelace formula, area and circumference of planes will run with the same complexity.\\

From this bound, if it takes time $t$ to compute the volume of a person $P$, it would take at most $4t$ to compute the volume of a person twice the size of $P$.\\

The above bounds are in no way a tight upper bound for the algorithm, as there have been many simplifying assumptions, such as $O(y) , O(p') \in O(p)$. This bound can most likely be tightened, perhaps to something akin to $O(y),O(p') \in O(\sqrt[3]{p})$ which would lower the running complexity to $O(\sqrt[3]{p}^2)$.\\

This analysis has not taken into account the cost of pulling each plane from the point cloud or sorting the data. These were omitted from the volume calculation analysis because they will be required for many operations, such as area and circumference calculation.\\

\subsection{Accessing the KD-tree and Sorting}
Querying the KD-tree to retrieve the necessary plane will take $O(p*log(p))$. These points will then need to be sorted so they are ordered in a circular fashion, taking $O(p'*log(p'))$ using quicksort. For quicksort, the worst case occurs when the algorithm consistently picks the worst pivot each iteration.\\

The probability of this happening is $O(\frac{1}{p^2})$, as such it is unlikely to occur in general use. In the case of PARSE, $p$ the order of $10^6$, such a large $p$ means the worst case is extremely unlikely to occur in practice. With the population of the UK being approximately $6 * 10^7$ \cite{UnitedKingdomofGreatBritain2011}, the entire population would have to be scanned 10,000 times before a worst case scenario is likely to have happened.\\
\section{Limb Circumference}

Limb circumference uses point cloud partitioning combined with the plane pulling algorithms associated with volume estimation that allow for the convex hull of planes in the respective area of the point cloud to be calculated and an approximate circumference returned.

\subsection{Partitioning the Point Cloud}

Partitioning of the point cloud so that planes could be taken for circumference calculation required translation of the inferred skeleton points tracked by the Kinect into Point Cloud space so that the appropriate bounds around each cross section of the point cloud could be taken. These bounds were defined for each of the respective feature points for limbs of interest. The encapsulation of the captured scan in the \texttt{PointCloud} data structure allowed a region of the point cloud to be extracted using the \texttt{range} function which took an (x,y,z) interval defined in point cloud space from the uniquely indexed underlying KD-Tree structure. A check was implemented when extracting the point cloud to ensure that sub regions were actually being extracted from the correct region. If the bounds were non-sensical (\textt{$x_{min} > x_{max}$}, for example) then the skeleton had not been tracked properly (as was often the case when it had inferred particular body parts during the scan incorrectly or had tracked an object that resembled a skeleton. The user was then asked to rescan to ensure an accurate capture. \\

The partitioning of the point cloud was fixed to the following local areas of the body; \emph{chest, shoulders, left arm, right arm, left leg, right leg, waist}. Any further partitioning of these limbs would have resulted in varied point clouds depending on the subject and high degree of noise from the increasing degradation in depth point quality at these levels.\\

\subsection{Corrective Constants}

\begin{figure}
\begin{center}
    \includegraphics[scale=0.6]{zscreenshots/limbdetails.png}
    \caption{Detailed view of limb circumference for the left arm of the Papaconstantinou subject}
\end{center}
\end{figure}


The requirement for corrective constants for the limb circumference measurement is as a result of the quality of the point clouds partitioned. The quality of the partitioned point clouds is dependent on the accuracy of the skeletal points that have been translated into point cloud space, if this translation is inaccurate than the partition will be poor or incorrect. Even if the quality of the point cloud partitioned well, the presence of clothes, the quality of the point cloud's affine ICP and the size of the patient all attribute to the need for correction to the calculated circumferences. In particular, while the mapping between the skeletal points and the point cloud is relatively accurate; for patients of an above average nature, it may be possible that the bounds specified do not fully capture the limb and thus underestimate the total circumference of a particular area of the body. These corrective constants are determined fully on a number of patients in the \emph{Testing} section of this report.

%complete
\section{Point Cloud Stitching}
\label{design:registration}
\label{design:sitching} %legacy, do not remove
\label{design:stitching}
The specification states that only a single scanner is to be used. This necessitates the utilisation of multiple scans and the stitching thereof into one unified data structure. This data structure can then be manipulated to produce volume estimation, as described in Section \ref{design:volume estimation}. The process of stitching these point clouds together to create one unified dataset is known as \emph{registration}\cite{Makadia2006} which has been described in Section \ref{research:registration}. The development of ideas leading to the final algorithm will be described in this section.  \\

\subsection{Input Data}
Depth maps of people (shells) are to be passed to the registration algorithm from a \emph{person isolation} subsystem. 

\subsection{Design Pattern}
\label{design:design pattern}
The various point cloud stitching algorithms all extend a class called ``Stitcher``. This makes it easy for new stitching algorithms to be deployed within the PARSE environment as only a single line of code needs to be modified in order to use a different stitching algorithm.\\

\subsection{Initial Experimentation}
Initially the point clouds were stitched by rotating successive scans 90\degree \ , increasing by a further 90\degree \ for each previous scan that has been taken, in an anticlockwise direction. The scan was then translated by a constant vector to produce a reasonable point cloud. This was able to produce reasonable results in individual cases but people come in a variety of ``depths`` and such an algorithm would only be useful in a world where everyone had constant "depths". \\

\subsection{Development of a more sophisticated approach: Bounding Boxes} 
\label{design:bounding boxes}
\label{design:bounding box}
While the Kinect is designed to produce depth information, it is only capable of scanning the visible surfaces of objects in front of the point in the 3D world in which it is situated. This has led to the coining of the term \emph{2.5D} to explain the limited information that the Kinect, and all single depth sensors, are able to gain about the world around them \cite{lu2006}. In theory this very limitation can be exploited to produce a reasonable, but by no means perfect, point cloud registration most cases. Within the context of this project this has come to be known as the \emph{Bounding Box} method of point cloud registration. \\ 

\subsubsection{Scanning Phase}
As with the initial experimentation, the bounding box method requires four scans; $S_1, S_2, S_3, S_4$. After each scan is obtained the patient is rotated by 90\degree \ in an anti clockwise direction. Each scan contains length and width information which is used in the processing phase. These scans are then packaged into a list data structure and sent to the bounding box method for processing. \\

\subsubsection{Processing Phase}
The processing phase consists of a number of simple rotations and translations. The subject was only rotated about the vector $(0, 1, 0)$ which significantly reduced the complexity of the operations. Table \ref{tab:bb processing} shows the rotations and translations that were performed for each scan. \\

%table 
\begin{table}
    \begin{tabular}{| p{0.8cm}| p{1.4cm} | p{4.9cm} | p{3.9cm} |}
    \hline
        Scan & Rotation & Rotation Axis & Translation  \\
        \hline
        $S_1$ & 0 \degree & $(0, 0, 0)$ & (0,0,0)  \\
        $S_2$ & 90 \degree & $(S_2(x_{min}), S_2(y_{min}), S_2(z_{min}))$ & $(0, 0, width)$  \\
        $S_3$ & 180 \degree & $(S_3(x_{min}), S_3(y_{min}), S_3(z_{min}))$ & $(S_3(width), 0, S_2(width))$  \\
        $S_4$ & 270 \degree & $(S_4(x_{min}), S_4(y_{min}), S_4(z_{min}))$ & $(S_3(width), 0, 0)$  \\
        \hline
    \end{tabular}   
    \caption{The processing that takes place on each data item for the Bounding Box method}
    \label{tab:bb processing}
\end{table} \\

\subsubsection{Challenges}
Despite the apparent successful stitching in many cases, the \emph{Bounding Box} method produced erratic results occasionally. Due to anatomical differences, male subjects tended tended to be stitched more accurately than female ones. This was also the case with obese subjects. This problem has been internally known as the \emph{Breast Problem} (Section \ref{testing:the breast problem}). \\

To address this; a more sophisticated registration method, the Iterative Cloud Positioning (ICP) algorithm (Discussed in Section \ref{res:icp}) was investigated. One of the big problems with the ICP algorithm, and other registration algorithms, is that they require some form of initial alignment which is then improved over time. This initial problem had already been potentially solved using the Bounding Box method and so it has been utilised part of a processing pipeline. \\

\subsection{Manual Alignment}
As discussed in the previous section, it is necessary that some form of initial alignment is made before most algorithms would be useful. To address the breast problem (Section \ref{testing:the breast problem}) it could be possible to offer some kind of manual alignment of the body panels. This, however, would potentially introduce unpredictable human error and would require real-time visualisation and modification of the point cloud data structures. Performing just four rotations and one rendering, as in the Bounding Box method \ref{design:bounding box}, takes several seconds to complete and therefore a new trade-off of image resolution vs display and processing hardware would be introduced. The manual alignment idea was ultimately abandoned due to the overriding objective of keeping the final product cheap in terms of hardware requirements, accurate and easy for the operator to use. \\

\begin{figure}[h!!]
    \label{fig:registration pipeline}
    \begin{center}
        \includegraphics[scale=0.6]{zscreenshots/reg-pipeline.png}
        \caption{A pipelined approach towards registration}
    \end{center}
\end{figure}

\subsection{Final design: Iterative Closest Point (ICP) algorithm}
So far it has been explained how a simple method that is the bounding box method is able to produce reasonably good results. In Section \ref{res:icp} we learnt that the gold standard for registration is the Iterative Closest Point method which requires some initial alignment. This alignment could be provided by the bounding box method. \\ 

Using the information that from experimentation and the research in Section \ref{res:icp} a pipelined approach to registration was developed using the three steps, explained in Figure \ref{fig:registration pipeline}.


\section{Markerless Recognition}

\subsection{Colour Search}
An application was designed, which would analyse the video feed from the Kinect. An initial, rudimentary, mechanism was used to highlight values in the output image on a binary basis: matches in white, the rest in black. A more efficient algorithm, if needed, would be put in place during any further development.\\
 
\begin{description}
\item[Capture] Frames captured from device
\item[Convert] Convert to HSL colour space, if required
\item[Search] Check very pixel value against target range
\item[Filter] Threshold data to remove noise, by keeping only pixels with a sufficient number of surrounding target pixels
\end{description} \\
 
Attaching a control to the original video feed allowed click-to-select and click-to-find functionality, which makes it significantly easier to obtain and then search for and refine the colour of an object.\\

\begin{description}
\item[Select] Click on video feed
\item[Search] Select search button
\item[Refine] Adjust colour components and variance
\end{description} \\

\begin{figure}
\begin{center}
\includegraphics[scale=0.2]{zscreenshots/screenshot_testing_rgb_orange}
\caption{The RGB colour search test environment}
\end{center}
\end{figure} \\

\subsection{SURF}
The SURF implementation was incorporated as a visualisation method in the main application, which allowed simple access to the Kinect and the possibility for further development later. Due to the computational complexity, it was not possible to run SURF in real time on the hardware available. Each SURF run takes a few seconds, which makes video or real-time testing non-viable. The SURF classifier was therefore tested only on static images, by creating a mechanism by which two frames were captured in a clear, guided process on-screen.\\

The default parameters were investigated, and the setup was deemed sufficient for the tests. Minor alterations were tested off-record, with little to report. \\

The program was configured to output a single image containing, on the right side, the target image with its features highlighted, and on the left side the input image with its features highlighted. Any correspondences between the two images' features are indicated with lines, and any target identified is indicated by a rectangle.\\

\begin{figure}
\begin{center}
\includegraphics[scale=0.25]{zscreenshots/SURF-Example.jpg}
\caption{An output image, with the found target indicated}
\end{center}
\end{figure}

\subsection{Haar}
The generation of the Haar classifier was to be fraught with complex procedures and intricate setup details. It took several weeks’ research and development to train a classifier at all, before testing could begin.\\

Limitations to this progress were numerous. For classifier use, the PCL library imports a classifier from a file, which is generated externally. To train the classifier, a precise setup must be created and then generated through use of an executable via an obscure command line process. Further, the training process requires large numbers of training images. The documentation suggested minimal numbers in the order of thousands; a quantity which would take quite some time to collect.\\

The first experiment was, therefore, to find the minimum number of samples required to find a simple object. An application was created to allow the import, classification, and export of images into the correct formats and directory structures required for the training process. A sample of fifty positive, classified images and fifty negative images was found to be sufficient to generate a classifier.\\

\subsection{Final Scan Process}
The final implementation, achieved after testing the tracking methodologies, used the colour space search in RGB format. We combined the RGB tracking with positioning components and a carefully designed workflow to create the final process. \\

\begin{itemize}
\item First, the operator and patient enter the field of view. Once the Kinect detects two skeletons, we search for the scanner using the RGB tracking mechanism.
\item Utilising the Kinect's skeletal data, the program searches for hands within a threshold distance of the scanner. Once one is found, that identifies that skeleton as the operator; and thus the other as the patient.
\item To capture the scan location, the operator need only hold the scanner still for a few seconds in order to trigger the capture routine. 
\end{itemize} \\

Looking ahead, the capture routine was engineered to fire an event within the C# runtime, which allows entities outside of the tracking class to be notified of the capture signal. This event could therefore be captured by some process intended to action data capture from a real sensing device. \\

\subsection{Coordinate Systems}
A key problem which was to be solved was the translation of the image-relative position of the sensor (in image coordinates) into real-world / skeletal coordinates. The coordinate systems do not align easily, nor exactly due to the physical offset in position between the depth and optical cameras. The Kinect API provides a set of coordinate space translations, but only into the RGB space, rather than from it. \\

To obfuscate matters further, the skeletal (real-world) coordinate system used by the Kinect uses ranges of -2.2 to +2.2 in the x axis, and -1.6 to 1.6 in the y axis.\\

The final solution used a basic trigonometric model, at fixed depth, because it was not possible to find the correlating depth value for a given RGB image pixel: \\

\begin{equation}
Real X = (((depth * 2 * tan(57) * X) / 640) - 2.2) * 1000
\label{Conversion of x coordinates from image into real space}
\end{equation} \\

\begin{equation}
Real Y = (((depth * 2 * tan(21.5) * (Y - 240)) / 480) + 1.05 ) * 1000
\label{Conversion of y coordinates from image into real space}
\end{equation} \\

\subsection{Skeleton-relative Positioning}
Following finding the location of the scanner in image coordinates, the location needs to be captured as a skeleton point relative to skeleton key points. For this purpose, the aim is to find the scanner location relative to the skeleton joints nearest to it.\\

The initial method for achieving this was implemented by using Pythagoras Theorem to calculate the distance of the scanner position relative to all joints. The joint that appeared to be the closest to the position was chosen and the scanner position was stored by using the difference between its coordinates and the coordinates of the closest joint. Even though this method appeared to be mostly adequate in our experiments, it was decided to implement some additional filtering of the joints before selecting the nearest one, in order to avoid errors that could potentially arise in situations where the body-type of the patient is very different to the ones that this method was being tested with.\\

The additions to the original method include finding the area around which the scan was made, from one of three main areas: legs, arms and chest. The division of the body into these areas is performed based on the 3 hip joints and the shoulders. Anything under the hip joints is classified as part of the legs and anything on the left and right of each of the left and right hips respectively is classified as arms. To confirm the arms, the shoulders are used as well. Following this division, if the scanner's location is on the chest area, the original method using Pythagoras Theorem will be applied, excluding all joints on the arms and legs, but finding the distance of the location from 3 joints instead of one, to improve accuracy of repositioning. If the scanner is on the legs or arms, the specific bone it is on will be found first, by comparing the position with the y of the joints of the legs or the x of the joints of the arms. Once that is determined and we have the 2 joints that make up the bone found, the distance of the scanner's coordinates from the coordinates of those joints is stored.\\

\subsection{Skeleton Identification}
Originally, it was intended that the Kinect skeletons of the operator and patient be intuitively identified by some measure independent the scan process, i.e. they should not have to do anything to identify themselves. This process was to be enabled by utilising the known locations of the scanner and the skeletons: the hand closest to the scanner (and within some threshold) must be that of the operator.\\

This mechanism proved to be problematic, however. The relatively poor reliability of the scanner tracking and of the hand tracking (from the Kinect API) meant that the process often took several seconds before the coordinates aligned sufficiently. A confirmation dialog was introduced during development, but the use of this caused yet more problems, as upon the operator's return to the frame they were allocated a new trackingID.\\

The final solution, although not ideal, was to have the operator stand to the left of the patient at the beginning of the scan process. This allowed for near-instant identification, saving up to thirty seconds against the previous method.\\

\section{External Libraries}
A number of external libraries have been used to provide data abstraction and optimised functionality. This section will explore details about their licensing and usage within the PARSE toolkit. \\

\subsection{Kinect SDK}
The Kinect SDK was the basis for the development of the project but the group only utilised a subset of it's functionality as our specialist point cloud reconstruction and volume estimation required access to the raw depth feed as necessitated for reconstructing the body. The SDK itself provided a means of accessing functions which allowed data to be mapped onto different coordinate spaces for the reconstruction of the point cloud. The Skeletal tracking technology also provided by the Kinect was used extensively in order to when a body is in view of the kinect for calibration, scanning, and markerless recognition purposes. It has also been used for the basis of isolation as it provided useful cues as to the appropriate level at which to perform a depth cut off. \\ 

\subsection{Math.NET}
A library called Math.NET was utilised extensively in the ICP portion of the solution to make up for the lack of support for data structures representing matrices and their associated operations within C# and the .NET framework \cite{mathdotnet}. The framework only contained basic operations such as additive and multiplicative operations and contained only a small subset of the operations that are available in an application such as MatLAB. Due to the lack of free matrix manipulation libraries Math.NET was used as a compromise and much of the matrix manipulation functionality was re-implemented in a specialised manner for this project. It is available under the MIT/X11 license which is a permissive licence allowing the usage of the software in both free and commercial applications as long as attribution is given \footnote{Fulltext available at: http://www.xfree86.org/3.3.6/COPYRIGHT2.html}. \\

\subsection{Helix3D}
Helix3D is a WPF based visualisation framework which adds a number of helper functions and viewport interactivity to the existing WPF3D framework \footnote{HelixLink}. Since it's inception, WPF3D has generally been considered a poor alternative to more established high performance frameworks such as OpenGL \cite{WpfPoor} but due to our language choice and the declining support of DirectX, WPF3D combined with Helix provided an adequate means of visualising the reconstructed point clouds and planes extracted from segmented point clouds after volume calculation and limb circumference determination. \\  

\subsection{WpfToolkit}
\label{design:wpf}

WpfToolkit added further functionality to the Windows Presentation Framework such as charting and visualisation for Markerless recognition/recall tasks. \\
\section{Interface Design}
\label{design:interface}

The interface was designed using the WPF framework. Each component of the user interface was written to support it's respective functionality and each UI unit is self contained. The use of a centralised class for dispatching events and handling core functionality means that data present in each functional UI unit of the system can be accessed via the relevant procedures in the CoreLoader. This allows the respective functions access to each others data and methods where appropriate.

\subsection{Intended Audience}

It is anticipated that this project will be used by researchers or medical practitioners on a initial patient examination level rather than on a diagnostic basis. Therefore the functionality of the system and the results it returns need to presented in a way which can be easily understood by both medical practitioners and the general public. This is as per the non-functional requirements facilitated through an adequate user interface.

\subsection{Component Design}

For the design of each component of the UI, the group agreed to follow the MVVM design pattern where the business and presentation logic of the system is separated from the technicalities of implementing the user interface. This meant that we could use the user interface to expose the data objects generated by visualisations or calculation of volumes or circumferences. WPF does not actually support the MVVM pattern of design due to all user interface components using an underlying XAML representation with access to data objects permitted by a data binding at run time. This required us designing the interface that would emulate such functionality. This was achieved by using separate user controls that would bind to each data object generated by the functionalities of the system.

Each user interface component uses a \texttt{\_Loader} convention to signify it's relation to the CoreLoader interface. The CoreLoader UI component is the parent class of all other UI components. This means child components can consist of other windows or custom user controls. It also reflects the architecture of the system. 

\subsubsection{CoreLoader}

This component is the owner of all other user interface components. This means that any global operations applied to this component affect all other currently open components. CoreLoader provides access to all the functions of the system.

\subsubsection{ScanLoader}

ScanLoader contains the viewer for visualising models representative of scanned pointclouds. ScanLoader allows for interaction with the model including panning and zoom as well as permitting aesthetic changes to the model such as colour or rendering.

\subsubsection{ViewLoader}

This component shows the raw data and overlay streams of the Kinect device. ViewLoader is capable of representing colour, depth and skeletal as well as feeds that incorporate additional post processing such as depth isolation or color isolation of a given subject.

\subsubsection{HistoryLoader}

HistoryLoader shows the results of any volume or limb circumference calculations carried out on a patient. These calculations will also be visualised depending on if planes need to be visualised for volumes. HistoryLoader also contains charting of results according to previous scans that the patient may have been subjected to in order to show changes in measurement for both limb circumference and volume.

\subsubsection{PatientLoader}

PatientLoader is a component that allows for the entry, editing and viewing of details related to the patient that is stored on the system database. 



\newpage
\section{Summary}
This section summarises the above design decisions.\\

\subsubsection{Person Isolation}
The depth and colour based isolation, described in \ref{design:depth based isolation} based on Kinect skeleton data was used over KDE and MOG. This route was taken because similar computation complexity of depth based isolation, the capability to isolate stationary people and the ease of implementation.\\

\subsubsection{Registration}
Point clouds will be stitched together using the Iterative Closest Point algorithm due to its proven good quality results over the past decades. This has been explained further in Section \ref{research:registration}. Initial alignment will assist the ICP algorithm using the bounding box methodology. These two steps should produce a point cloud that can be used by the calculations explained in this section to a satisfactory degree of accuracy. \\

\subsubsection{Volume Estimation}
The volume estimation continues to build on the SSP methodologies originally researched. The algorithm will split the point cloud into planes and calculate the area of each, using the Shoelace formula. Each area will then be multiplied by the distance between the planes to give a volume. These volumes will then be summed to give the final output. Not only is the volume calculated, but the circumference of each plane used has been calculated to provide additional information.\\

\subsubsection{Markerless Recognition}
Testing will reveal the effectiveness of three image search methodologies, RGB colour space search, SURF, and Haar-based classification. The most effective will be used as the basis for the scanner tracking. From there, translation to the real coordinate space and use of the Kinect's skeleton tracking API should allow the calculation of skeleton-relative positions.
