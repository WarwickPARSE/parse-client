\chapter{Testing and Analysis}
\label{testing}

This section of the report outlines the testing necessary in order to evaluate the accuracy and robustness of the volume reconstruction, volume and limb circumference calculation and the recording and recall of medical scanner postioning on the human body. This testing informs our evaluation of the project. It is the intention of the group to demonstrate that this non-invasive and 'hands-free' Kinect based methodology provides at least a good approximation of measurements achieved through more accurate means of testing. 

\section{Test Plan}

We prepared a test plan at the start of the project which acted as motivation for defining the internal deliverable deadlines. Along with this test plan, we have outlined a strategy by which to test each component of the system thoroughly.

\subsection{Test Strategy}

\label{test strategy}

Each functional aspect of the system has been unit tested by their respective group member with their results and analysis presented in each section of this chapter. The unit tests are concerned with the operation of: \emph{\bf{person isolation, registration, volume calculation, limb circumference, image recognition and database functionality.}} Integration testing is then performed and the functionality of respective modules tested against one another, for example, the scanning of patients being used along side the database engine and visualisation module. System-wide testing is then examined with a number of test cases that aim to evaluate the toolkit as one coherent system.

\subsection{Test Controls}

The environment in which subjects are to be captured must be as consistent as possible. As such the group have defined a number of controls that is imposed upon persons being scanned that will return the results with reasonable consistency and accuracy.\\

\emph{\bf{Distance from Kinect}} - In order to capture the entire body and to ensure consistent positioning between volume scans, the subject will stand at approximately 2.2m away from the Kinect. This distance was chosen, despite the fact it is outside of the Kinect's \emph{recommend play space} \cite{xbox2010}, as it is far enough away to accommodate subjects who are at the upper bound of test patients in terms of height without compromising too much on the quality of the captured depth map associated with the subject. Our configuration means that for each side of the patient that is scanned, this distance is maintained. The reconstruction process then does not need to take into account global scaling of the point cloud when translating and rotating it into position. \\

\emph{\bf{Position of the Kinect}} - In order to ensure fair test conditions, when a volume scan is taken the Kinect must be positioned 71cm off the floor. The scanning process will automatically change the elevation of the Kinect to 0 when a new scan is initiated.\\

\emph{\bf{Lighting Conditions}} - Scans were taken in the same environment where lighting was adequate but not too bright so as to interfere with the CMOS sensor's ability to track different levels of depth in the scene. Lighting was also important for the purposes of skeletal tracking as the SDK's algorithms relied on tracking the human body in RGB space for this purpose. \\

\emph{\bf{Occlusion and Noise}} - During development we found that the Kinect was particularily sensitive to noise leading to distorted depth maps where lighting was in adequate or there were objects in the foreground of the subject to be scanned. Occlusion was also an issue if the background was particularly crowded. We aimed to minimise the amount of noise and foreground interference by removing objects in the scene that may have otherwise interfered with the scan process. \\

\emph{\bf{Scan Process}} - The scan process itself is controlled by how it has been implemented. The subject is given 10 seconds to position themselves between each capture and by maintaining the ideal depth; a reasonably accurate scan can be captured permitting other influencing factors. \\

\emph{\bf{Clothing}} - Clothing presents an issue where the volume measurement or limb circumference measurement may be overly affected by a subject that is wearing baggy clothes or clothes that enlarge certain areas of the body. This has been identified as problematic in other 3D body scanning applications where the presence of clothing has interfered with the landmarking of particular areas of the body \cite{Dekker1999}. For the purposes of scanning we have requested that each subject remove any clothing apart from T-shirts and trousers in order to gain more accurate volume measurements and minimise subject discomfort. \\

\subsection{Test Schedule}

A test schedule was devised so that adequate testing of the volume estimation, limb circumference and marker-less tracking components of the system could be performed and changes made to them in order to ensure correct working functionality and accuracy when the system is eventually delivered to the customer. 

\begin{figure}
\centering
\begin{tabular}[htb]{| l | l | l |}
    \hline
    Unit Test & Date Started & Date Completed \\ \hline \hline
    Volume Estimation & 23/02/13 & 01/04/13 \\ \hline
    Limb Circumference & 25/03/13 & 23/04/13 \\ \hline
    Markerless Recognition & 01/04/13 & 20/04/13 \\ \hline
    Database Integration & 04/04/13 & 21/04/13 \\ \hline
    Unit Integration & 01/03/13 & 20/04/13 \\ \hline
    System Functionality & 05/04/13 & 21/04/13 \\ \hline
\end{tabular}

\caption{Test plan for the PARSE System}
\end{figure}


\section{Person Isolation}
\label{design:person isolation}
The group decided to make use of the Kinect's abilty to generate a skeleton frame associated with a person to aid in isolation. This decision meant the group stepped away from the generic computer vision algorithms discussed in Section \ref{person_isolation:specific algorithms}, such as KDE and MOG.\\

Two possible methods of person isolation were designed by the group. The first method made use of the Kinect colour stream whereas the second used the Kinect depth stream. Both methods described operate on a per pixel basis, have a complexity of $O(n)$ where $n$ is the number of pixels in a frame and are theoretically capable of isolation stationary people, which the reserached methods may not be able to do.\\

\subsection{Colour Based Isolation}
\label{design:colour based isolation}
In this algorithm, the pre processing is handled by the Kinect API, converting the raw infra red depth data into a byte array. The foreground mask is calculated using the Kinect API to determine whether a colour pixel is associated with a detected skeleton. If a pixel is associated with a skeleton, the pixel is a foreground pixel. And conversely, an unassociated pixel is a background pixel. There is no data validation in this algorithm.\\

At this stage it was expected that colour based isolation would preform well, but it was unknown if a point cloud could be constructed using this method, as the colour data contains no depth.\\

\subsection{Depth Based Isolation}
\label{design:depth based isolation}

In this algorithm, the pre processing is again handled by the Kinect API, converting the raw infra red depth data into a byte array. This algorithm then make use of the skeleton to determine the approximate depth of the person. Any pixel whose depth value is outside a delta of the skeleton's depth would be considered a background pixel. Cutting off based on depth alone is not enough, as doing so would leave a ring of equidistant points in-line with the person.\\

To eliminate this ring, the positions of left and right most point of the person (i.e. the HandLeft joint and the HandRight joint) would also be used for cut off and anything outside of this range would also be classified as a background pixel and discarded. All other pixels would be considered foreground and again there is no data validation phase.\\ 

This method would leave a square of floor under the person, but at the design stage it was hoped this could be removed at the point cloud level. Whilst a depth based cut off may not be as effective at removing all the miscellaneous non-person data, i.e. the floor, it may be better suited to creating a point cloud than colour based isolation.\\
\section{Registration}
\label{testing:registration}
This section deals with the testing of the registration functionality, as described in Section \ref{design:registration}. 

\subsection{Methodology}
Testing this sub-system was not a trivial exercise due to the interdependence with volume estimation. That is, correct volume estimation requires correct registration but the correctness of registration can only be measured by observing correct values of volume estimation.\\

The best way of determining the accuracy of stitching is by analying the deviance of the Toolkit Volume Measurement from the Gold Standard Volume measurement, which can be read in Section \ref{testing:vol est}. The first data point, Bernard, is a control and the error is irrelevant. \\
\section{Volume Estimation}
\label{volume estimation}
\label{testing:vol est} %delete this and die 
This section details the testing of the volume estimation algorithm as well as touching on the testing of the circumference estimation.

\subsection{Determining the Transform Constant Using Volume}
\label{determining the transform constant}
Initially, the transform constant was to be calculated in terms of volume. The approach planned was for many people to be scanned and their minimum bounding box calculated two ways. Firstly the box would calculated using the PARSE toolkit to give the box's volume in PCS. Secondly, the volume of the box would be calculated in RWS by measuring the height, width and breadth of the person. These figures would then be divided to determine the transform constant as in equation \ref{testing: calculating the transform constant}.\\

\begin{equation}
\frac{Volume_{RWS}}{Volume_{PCS}} = Transform Constant
\label{testing: calculating the transform constant}
\end{equation}\\

However, delays in the stitching algorithm for the four individual scans were holding up the determination of the transform constant via this method and therefore the testing of the overall volume estimation. As such, a different approach was taken. 

\subsection{Determining the Transform Constant Using Height}
\label{determining the transform constant using height}
Instead of measuring a bounding box two ways, the transform constant would be determined in terms of height, as in equation \ref{testing: calculating the transform constant using height}. This method would only required un-stitched scans. Using height resulted in the need to cube the constant to transform 3-dimensional quantities such as volume.\\

\begin{equation}
\frac{Height_{RWS}}{Height_{PCS}} = Transform Constant
\label{testing: calculating the transform constant using height}
\end{equation}

In order to determine the transform constant, twenty two subjects were scanned and their height measured in both point cloud units (PCU) using the tool kit and in meters using a tape measure. Using the data in Table \ref{testing: table of data used to calculate the transform constant}, a transform constant was calculated for each subject.\\

\begin{table}[!htb]
\begin{center}
  \begin{tabular}{| l | p{3cm} | p{2cm} | p{2cm} |}
    \hline
    Person & $Height_{RWS}$ ($m$) & $Height_{PCS}$ & Transform Constant \\ \hline
    Adams & 1.88 & 2.29 & 0.82\\ \hline
    Alissa & 1.86 & 2.27 & 0.82 \\ \hline
    Anonymous 1 & 1.82 & 2.28 & 0.80\\ \hline
    Anonymous 2 & 1.76 & 2.24 & 0.78\\ \hline
    Archbold & 1.82 & 2.26 & 0.80\\ \hline
    Bradbury & 1.72 & 2.11 & 0.82\\ \hline
    Corbett & 1.94 & 2.29 & 0.85\\ \hline
    Fletcher & 1.80 & 2.25 & 0.80\\ \hline
    Getley & 1.85 & 2.26 & 0.82\\ \hline
    Griffiths & 1.84 & 2.27 & 0.81\\ \hline
    Janssens & 1.85 & 2.26 & 0.82\\ \hline
    Jin	& 1.81 & 2.24 & 0.81\\ \hline
    Kozlowska & 1.64 & 2.09 & 0.78\\ \hline
    Marshall & 1.73 & 2.12 & 0.81\\ \hline
    McCutcheon & 1.88 & 2.29 & 0.82\\ \hline
    Mermet & 1.70 & 2.11 & 0.80\\ \hline
    Page & 1.82 & 2.28 & 0.80\\ \hline
    Papaconstantinou & 1.69 & 2.10 & 0.81\\ \hline
    Richardson & 1.58 & 2.00 & 0.79\\ \hline
    Rayment & 1.92 & 2.29 & 0.84\\ \hline
    Taplin & 1.62 & 2.08 & 0.78\\ \hline
    Wicks &1.90 & 2.29 & 0.83\\ \hline
  \end{tabular}
\end{center}
\caption{Table of data used to calculate the Transform Constant (to two decimal places)}
\label{testing: table of data used to calculate the transform constant}
\end{table}

From these twenty two transform constants, an average transform constant was calculated to be 0.81 (to two decimal places). 

\subsection{Height Estimation}
\label{height estimation}
Using an average transform constant is a \emph{one size fits all} approach and will introduce errors into the height estimation. These errors are expected to be related to the distance a subject is from the average height and was quantified by predicting the height of a subject as per equation \ref{predicting real world height}.\\

\begin{equation}
Height_{PCS} * Average Transform Constant = Height_{RWS}
\label{predicting real world height}
\end{equation}

Table \ref{errors in determining a subjects height} shows for each subject, their absolute distance from the average sample height (ADFAH), the predicted height and the error in this prediction when compared to the subjects actual height in meters.\\ 

\begin{table}[!htb]
\begin{center}
  \begin{tabular}{| l | p{2.5cm} | p{2cm} | p{2cm} |}
    \hline
    Person & ADFAH (m) & Predicted Height (m) & Error (\%) \\ \hline
    Alissa & 0.07 & 1.84 & 1.16\\ \hline
    Anonymous 1 & 0.03 & 1.85 & 1.60\\ \hline
    Anonymous 2	& 0.03 & 1.82 & 3.14\\ \hline
    Archbold & 0.03 & 1.83 & 0.61\\ \hline
    Bradbury & 0.07 & 1.71 & 0.82\\ \hline
    Corbett	& 0.15 & 1.85 & 4.80\\ \hline
    Getley & 0.06 & 1.83 & 0.98\\ \hline
    Fletcher & 0.01 & 1.82 & 1.18\\ \hline
    Griffiths & 0.05 & 1.83 & 0.28\\ \hline
    Janssens & 0.06 & 1.83 & 0.98\\ \hline
    Jin & 0.02 & 1.82 & 0.39\\ \hline
    Kozlowska & 0.15 & 1.69 & 3.10\\ \hline
    Marshall & 0.06 & 1.72 & 0.63\\ \hline
    McCutcheon & 0.09 & 1.85 & 1.56\\ \hline
    Mermet & 0.09 & 1.71 & 0.61\\ \hline
    Page & 0.03 & 1.85 & 1.61\\ \hline
    Adams & 0.09 & 1.85 & 1.38\\ \hline
    Richardson & 0.21 & 1.62 & 2.34\\ \hline
    Papaconstantinou & 0.10 & 1.70 & 0.46\\ \hline
    Rayment & 0.13 & 1.85 & 3.75\\ \hline
    Taplin & 0.17 & 1.69 & 3.89\\ \hline
    Wicks & 0.11 & 1.85 & 2.58\\ \hline
  \end{tabular}
\end{center}
\caption{Errors in determining a subjects height}
\label{errors in determining a subjects height}
\end{table}

It was suspected that error was related to the subjects absolute distance from average height. Therefore, error has been plotted against this distance in Figure \ref{distance vs error}.\\

\begin{figure}[!htb]
\begin{center}
\includegraphics[scale=0.8]{images/distvserror} 
\end{center}
\caption{Distance vs Error}
\label{distance vs error}
\end{figure} 

ADFAH and error have a Pearson product-moment correlation coefficient of 0.61, suggesting a medium strength correlation \cite{cohen88,buda2011} between the two quantities. This suggests the averaging approach to determining the transform constant has the limitation that a subject must be average height, in terms of the sample used to determine the TC, in order for the most accurate result.\\

The errors in predicted height from the test subjects can be used to determine a approximate error range for estimating height of a subject between 1.58m and 1.94m. This range is the taken from the real world heights in Figure \ref{testing: table of data used to calculate the transform constant}. The average as well as the standard deviation error were determined as in Table \ref{average and standard deviation of error}. From these values, the height estimation could be said to have an accuracy of $\pm$4.25\%. This value is based on the fact that height is often assumed to be normally distributed \cite{chali1995} and ~95\% of the values of a normal distribution typically lie within 2 standard deviations of the mean value \cite{pukelsheim1994}.\\

\begin{table}[!htb]
\begin{center}
  \begin{tabular}{| l | l |}
    \hline
    Average Error (\%) & 1.72\\ \hline
    Standard Deviation of Error & 1.26\\ \hline
  \end{tabular}
\end{center}
\caption{Average and standard deviation of error}
\label{average and standard deviation of error}
\end{table}

The above determination of the transform constant allowed for the volume estimation to be tested on scans of a sufficient quality that the initial Bounding Box method of stitching aligned them near perfectly, as was the case for the Corbett data set, see Figure \ref{fig:the_corbett_data_set}. This testing is elaborated on in Section \ref{testing: volume calculation}.\\

\subsection{Errors in Height Measurement}
Calculating the transform constant in the manner described in Figure \ref{testing: calculating the transform constant using height} is not without drawbacks. The main drawback is that the real world measurements must be calculated by hand, using a tape measure. Such a process is inherently error prone, as highlighted in Section \ref{spec:motivation}. However it was hoped that, by taking as large a sample as possible and averaging the transform constants, such errors would be eliminated or at least minimised.\\

\subsection{Consistency of Height Measurement}
Whilst the one time error of the system is important to quantify, the consistency is also of note. To quantify this, the Corbett subject was repeatedly scanned the height of the subject measure. Using the data from Figure \ref{testing: table of data used to calculate the transform constant} as the gold standard, the average error over twenty scans was 2\% and the maximum error of 4\%, based on the standard deviation \cite{pukelsheim1994}.\\

\subsection{Plane Retrieval}
This section details the testing of the method to retrieve a plane of points, all at a similar height, from a point cloud was tested by outputting the retrieved points x and z co-ordinates to a .csv file and plotting them using Excel. Using Excel meant that the retrieval method could be tested outside the visualisation, which was at this stage untested. For testing purposes we took a slice from the approximate middle, height wise, of the Wilkinson data set, visualised in Figure \ref{fig:the wilkinson data set, visualised with the bounding box stitching method}.\\

\begin{figure}[!htb]
\begin{center}
\includegraphics[scale=0.4]{images/wilko1} 
\end{center}
\caption{The Wilkinson data set, visualised with the Bounding Box stitching method.}
\label{fig:the wilkinson data set, visualised with the bounding box stitching method}
\end{figure}

Through Excel, the group was able to determine the middle plane from the Wilkinson data set was being correctly retrieved, see Figure \ref{fig:the middle plane of the Wilkinson data set, visualised in excel.}. Also, the Excel visualisation was a gold standard that informed the verification of the PARSE toolkit visualisation, see Figure \ref{fig:The middle plane of the wilkinson data set, visualised in the parse toolkit.}.\\ 

\begin{figure}[!htb]
\begin{center}
\includegraphics[scale=0.4]{images/wilko2} 
\end{center}
\caption{The middle plane of the Wilkinson data set, visualised in Excel.}
\label{fig:the middle plane of the Wilkinson data set, visualised in excel.}
\end{figure}

\begin{figure}[!htb]
\begin{center}
\includegraphics[scale=0.4]{images/wilko3} 
\end{center}
\caption{The middle plane of the Wilkinson data set, visualised in the PARSE toolkit.}
\label{fig:The middle plane of the wilkinson data set, visualised in the parse toolkit.}
\end{figure}

\subsection{Using a Bath to Calculate Gold Standards for Volume}
In order to test the accuracy of the toolkit's volume estimation, a gold standard would be required. This volume was achieved by calculating the volume of water a person displaced \cite{katch1967}. This was achieved by filling the bath to a level and marking the level. The subject then sits in the bath, fully submerged, and a new mark made. The person then exits the bath after all the water has dripped off their body. Using a jug \cite{beynon1989} of known volume, the water level is increased until the level meets the second mark. From the number of jugs used, a gold standard volume can be determined. Table \ref{the volume of group members} shows these gold standards. Whilst they are being treated as true values, in reality it is unlikely they are. Reasons for this will be discussed in later in Section \ref{errors in volume measurement}.\\

\begin{table}[!htb]
\begin{center}
  \begin{tabular}{| l | l |}
    \hline
    Name (\%) & Volume ($m^3$)\\ \hline
    Bernard & 0.058\\ \hline
    Nathan & 0.062\\ \hline
    Greg & 0.064\\ \hline
  \end{tabular}
\end{center}
\caption{The volume of group members}
\label{the volume of group members}
\end{table}

\subsection{Volume Calculation}
\label{testing: volume calculation}
From the values in Table \ref{average and standard deviation of error}, hypothetical average error and the corresponding standard deviation can be approximated by cubing those values. Table \ref{hypothetical average and standard deviation of error} shows these hypothetical values. Hence, the volume estimation could be said to have an accuracy of between $\pm$1.06\% and $\pm$9.12\%. The Bod Pod has been shown to have similar error ratings \cite{fields2001,collins2004}.\\

\begin{table}[!htb]
\begin{center}
  \begin{tabular}{| l | l |}
    \hline
    Average Error (\%) & 5.09\\ \hline
    Standard Deviation of Error & 2.01\\ \hline
  \end{tabular}
\end{center}
\caption{Hypothetical average and standard deviation of error of volume estimation}
\label{hypothetical average and standard deviation of error}
\end{table}

Using the toolkit, the volume of group members were calculated and are displayed in Table \ref{the volume of group members measured by the tool kit}. The volume of the Corbett subject was averaged over four scans.\\

\begin{table}[!htb]
\begin{center}
  \begin{tabular}{| l | l | l |}
    \hline
    Name (\%) & Volume ($m^3$) & Error (\%)\\ \hline
    Bernard & 0.07
 & 20.69\\ \hline
    Nathan & 0.075 & 20.97\\ \hline
    Greg & 0.0816 & 27.50\\ \hline
  \end{tabular}
\end{center}
\caption{The volume of group members measured by the tool kit}
\label{the volume of group members measured by the tool kit}
\end{table}

From Table \ref{the volume of group members measured by the tool kit}, the volume estimation could be said to have an average error measure of 23.05\%.

\subsection{Errors in Volume Measurement}
\label{errors in volume measurement}
Clothing can artificially inflate a subjects volume when scanned \cite{shafer2008}. It must be noted that the gold standard for a subject was determined whilst the subject was in a state of undress, whereas the scan were taken in a public work area hence clothes were worn, this may be the reason for the volume estimation consistently overestimating by approximately 20\%. A better measure to compare volumes and comment on the estimations accuracy may be ``the Bernard", defined in \ref{the bernard defined}. Using the Bernard, the gold standard volumes and the volumes calculated by the toolkit can be expressed as in Table \ref{The volume of group members in bernards}.\\

\begin{figure}[!htb]
\begin{center}
  \textit{Definition: 1 Bernard is the volume of Bernard Sexton, as calculated by the method provided by the context.}
\end{center}
\caption{The Bernard defined}
\label{the bernard defined}
\end{figure}

\begin{table}[!htb]
\begin{center}
  \begin{tabular}{| l | p{4cm} | p{3cm} | p{2cm} |}
    \hline
    Name (\%) & Gold Standard Volume (Bernards) & Toolkit Volume (Bernards) & Error (\%)\\ \hline
    Bernard & 1.00 & 1.00 & 0.00\\ \hline
    Nathan & 1.07 & 1.07 & 0.23\\ \hline
    Greg & 1.10 & 1.17 & 5.64\\ \hline
  \end{tabular}
\end{center}
\caption{The volume of group members in Bernards}
\label{The volume of group members in bernards}
\end{table}

Hence, the ratios of volumes between the three group members are fairly consistent when measured with both the gold standard bath tub and the toolkit, with an average error of 1.96\%. The majority of this error is associated with the Corbett subject who, as shown in Section \ref{height estimation}, is far away from the average height, so 4\% of the 5.64\% may be caused by his above average height. Excluding this height caused error would bring the average error of the ratios down to 0.62\%, but more test subjects are needed before any real conclusions can be made.\\

When calculating the gold standards factors such as water dripping, skin absorption and surface evaporation can all introduce errors. If the subject is not completely dry when they exit the bath the volume will be marginally overestimated. The skin of the subject may also absorb some of the water and whilst the measurements are being taken some of the water may evaporate again causing the volume to be overestimated. The jugs used to measure the quantity of water returned to the bath only give approximate measurements \cite{ikea,pyrex}. The refilling of the bath via these jugs is also vulnerable to parallax errors. Noisy scan data will also introduce random error into the estimations.\\

\subsection{Circumference Measurement and Errors}
\label{errors in circumference measurement}
In order to test the circumference measurements, the Corbett subject's circumference was measured with a tape measure at 0.49m, 0.97m and 1.46m corresponding to planes 45, 30 and 15. The results are reproduced in Table \ref{the circumference of the Corbett subject measured with the tool kit} and were treated as gold standards. The tool kit was then used to measures these circumferences using the same four scans as in Table \ref{the volume of group members measured by the tool kit}. The results are reproduced in Table \ref{the circumference of the Corbett subject measured with the tool kit}.\\

\begin{table}[!htb]
\begin{center}
  \begin{tabular}{| l | p{3cm} | p{3cm} | p{2cm} |}
    \hline
    Height (m) & Gold Standard Circumference (m) & Average Circumference (m) & Error (\%)\\ \hline
    0.49 & 0.62 & 0.56 & 10.14\\ \hline
    0.97 & 0.88 & 0.87 & 1.60\\ \hline
    1.46 & 1.06 & 0.91 & 14.41\\ \hline
  \end{tabular}
\end{center}
\caption{The circumference of the Corbett subject measured with the tool kit.}
\label{the circumference of the Corbett subject measured with the tool kit}
\end{table}

As previously mentioned, 4\% of this error may be caused by the Corbett subject being far away from the average height. As such, the circumference estimation could be said to be accurate to within $\pm$10\%. Clothing can increase the error on a circumference measure, a t-shirt sleeve suitable positioned can add 10cm to the circumference value so the true error rating may be less than this.\\

\section{The Breast Problem}
\label{testing:the breast problem}
\emph{The Breast Problem}  is an umbrella term coined to describe the many problems caused by the anatomical differences between the genders. An example of such a problem is the degree of accuracy when stitching with the Bounding Box method.\\

Another example is the errors when measuring circumferences caused by bras. It has already been shown that clothing can have a big impact on the measurement of circumferences in Section \ref{errors in circumference measurement}. If, for example, a measurement is needed at about breast height, the type of bra the patient is wearing may affect the data returned. An underwired bra may increase the circumference at this height or expose a previously occluded area of ``underboob". This issue has been previously noted by those determining the accuracy of the BOD POD \cite{shafer2008}.\\
\section{Limb Circumference}
\label{testing:limb circumference}
This section details the testing of the limb circumference algorithms. Circumference testing in the context of volume estimation is discussed and the equivalent accuracy of partitioned plane circumference estimation examined to validate whether such practice produces a reliable means for estimating limb size.

\subsection{Calculating Limb Bounds}

The 20 feature points provided by the skeleton are used to simplify the bounding of by referencing the minimal and maximal feature points of the interested limb. An assumption is made that the detected skeletal positions are reasonably aligned with the respective limb locations captured by the point cloud. \\

\begin{table}[!htb]
\begin{center}
  \begin{tabular}{| l | p{4cm} | r | r | r | r |}
    \hline
    Limb & Feature Points & $X_{min}$ & $X_{max}$ & $Y_{min}$ & $Y_{max}$ \\ \hline 
    Left Arm & ShoulderLeft, WristLeft & 0.250 & 0.267 & 0.0779 & 0.778  \\ \hline
    Right Arm & ShoulderRight, WristRight & -0.191 & -0.1652 & 0.03 & 0.808  \\ \hline
    Shoulders & ShoulderRight, ShoulderLeft & -0.61 & 0.251 & 0.462 & 0.947 \\ \hline
    Waist & HipRight, HipLeft, HipCenter & -0.053 & 0.138 & 0.270 & 0.371  \\ \hline
    Right Leg & HipCenter, HipRight, KneeRight, FootRight & -0.051 & 0.040 & -0.523 & 0.280  \\ \hline
    Left Leg & HipCenter, HipLeft, KneeLeft, FootLeft & 0.045 & 0.131 & -0.519 & 0.270  \\ \hline
  \end{tabular}
\end{center}
\caption{Feature points for each limb as recorded by Corbett Subject}
\label{testing: feature points used for each limb}
\end{table}

These limbs bounds are determined based on the anatomical reference points that the Kinect Skeleton SDK attaches to the tracked skeleton when undergoing a scanning procedure. The bounding is identified when the person is facing the front as the SDK is only capable of inferring skeletal cues when the subject is in a frontal configuration. The depth of the subject is defined in terms of the stitched point cloud where the minimum depth is recorded at the front and the maximal depth offset against the rear scan.

\subsection{Identifying Transform Constants}

As discussed in Section \ref{volume estimation}, there is a need for identifying transform constants between the circumferences calculated in terms of point cloud units and real-world and meaningful SI units. Using the height transform constant for the purposes of scaling circumference still provides a valid basis for accurate limb measurement as there is a broadly positive correlation between the height of an individual and in the cited cases, their arm circumference \cite{Todorovic2003}. However, on a local limb circumference measurement level, factors such as clothing, point cloud quality and the stability during of the person during scanning influences the integrity and accuracy of the planes extracted from the partitioned point clouds. \\

After bounding the limbs, the circumference of the subsampled planes is calculated. This circumference value represents the circumference with respect to the point cloud coordinate system and even when converted to standard units, the value is comparably small due to the lack of associated scale associated with it. A number of experiments have been carried out on subjects in order to determine these scale factors across a fairly representative spread of size and shape. The transform constants have been established from a subsample of the original sample of test subjects and classified according to below average, average and above average weight and overall size. \emph{Below Average} is determined as below 60kg and/or less than 1.6m in height. \emph{Average} is determined as between 60kg and 90kg and/or between 1.6 and 1.9m in height. \emph{Above Average} is determined as over 90kg or over 1.9m in height.  $Rw$ refers to the raw circumference as determined from calculation in the point cloud space, $Ac$ refers to the circumference as defined by spring tape measurement with a constant calculated from the difference.

\begin{table}[!htb]
\begin{center}
\begin{tabular}{| l | l | l | l | l |}
\hline
Subject & Limb & $Circum_{Rw}$ & $Circum_{Ac}$ & Transform Constant \\ \hline
    Below Average & Left Arm & 21.341 & 25.8 & 1.21 \\ \hline
    & Right Arm & 22.34 & 23.4 & 1.05\\ \hline
    & Shoulders & 57.22 & 82 & 1.43\\ \hline
    & Chest & 49.3 & 80 & 1.62\\ \hline
    & Waist & 11.3 & 75 & 6.63\\ \hline
    & Left Leg & 9.2 & 35.6 & 3.89\\ \hline
    & Right Leg & 12.3 & 36.5 & 2.96\\ \hline \hline
    Average & Left Arm & 23.195 & 25 & 1.07 \\ \hline
    & Right Arm & 24.12 & 25.5 & 1.05 \\ \hline
    & Shoulders & 64.39 & 92 & 1.41\\ \hline
    & Chest & 61.19 & 91.5 & 1.42\\ \hline
    & Waist & 14.82 & 88 & 5.94\\ \hline
    & Left Leg & 13.05 & 40.5 & 3.10\\ \hline
    & Right Leg & 14.9 & 44.0 & 2.95\\ \hline \hline
    Above Average & Left Arm & 35.443 & 32.5 & 0.91\\ \hline
    & Right Arm & 34.53 & 31 & 0.89\\ \hline
    & Shoulders & 68.123 & 104.5 & 1.53\\ \hline
    & Chest & 67.980 & 99.5 & 1.46\\ \hline
    & Waist & 22.3 & 101 & 4.529\\ \hline
    & Left Leg & 16.501 & 56 & 3.39\\ \hline
    & Right Leg & 16.091 & 56.5 & 3.51 \\ \hline \hline
\end{tabular}
\end{center}
\caption{Transform constants calculated for the required correction of different subject circumferences based on spring tape measurements.}
\label{testing: transform constants}
\end{table}

\subsection{Errors and Limb Circumference}

As shown in each person classification above, there is in some cases, a requirement for significant correction of the the recorded point cloud circumference when compared to the circumferences recorded using spring tape measurement. The need for this correction is particularly profound around the lower extremities of the body with the worst case being recorded in the lower than average person on the waist necessitating a 6.5 scaling factor to align the toolkit's measurement with what is realistically expected. Through further testing it was ascertained that the patient scanning configuration where the legs were positioned close together was not an optimal configuration for mapping the skeletal co-ordinates into point cloud space. This was due to the Kinect API's inability to infer both the left and right legs properly and inferring the waist from the straight arm configuration in the upper extremities. In some cases this lead to a deviation of $[-0.5,0.5]$ from the original assumed centre of the limb in point cloud space from the inaccurate recording of limb positioning from the Kinect SDK. \\

Obviously further refinements were needed to the point cloud partitioning bearing in mind the sometimes erroneous or imprecise mapping of the skeletal world coordinates to point cloud space. When refinements were added so that tighter bounds were specified over particular limbs, a marked improvement was seen in the circumference calculation of upper extremities such that an average error of $10\%/15\%$ was observed. There was also an improvement on waist measurements due to the incorrect bounding method that was originally applied. Given these refinements, an average error over upper extremities (in the present results, the left arm) ranged between $2.0\% - 17\%$ and lower extremities varied between $10\%-36.6\%$. \\

\begin{table}[!htb]
\begin{center}
  \begin{tabular}{| l | r | r | r | r | r | r |}
    \hline
    Person & $Arm_{Ac}$ & $Waist_{Ac}$ & $Arm_{Ki}$ & $Waist_{Ki}$ & $Error_{arm}$ & $Error_{waist}$  \\ \hline
    Corbett & 27 & 84 & 25.9 & 69.33 & 4.2\% & 21.1\% \\ \hline
    Eddie & 27.2 & 81.9 & 23.1 & 59.3 & 17\% & 36\% \\ \hline
    Page & 24.5 & 82 & 24 & 72 & 2.01\% & 13.8\% \\ \hline
    Papas & 23.5 & 89 & 26.13 & 65.13 & 11.1\% & 36.6\% \\ \hline
    Rodolis & 25 & 91 & 29 & 73.4 & 16\% & 23.9\% \\ \hline
    Sexton & 26.4 & 83 & 24.80 & 72.1 & 6.5\% & 15.1\%\\ \hline
  \end{tabular}
\end{center}
\caption{Errors measured over 6 test subjects.}
\label{testing: table of data for limb transform constants}
\end{table}

\subsection{Sources of error}

There are a number of potential error sources that result in the error observed in these results and some have already been alluded to. There are 3 primary areas where error is introduced into the limb circumference calculation:

\begin{enumerate}
    \item \emph{Poor point cloud registration}; where the stitching algorithms for registering each of the 4 point cloud scans returned poor results, the circumference measurements were affected by the lack of the required ring of points needed for circumference calculation so the gift wrapping algorithm would often calculate the circumference for a subset of possible points returning results that were infeasibly small as a result.
    \item \emph{Clothing}; normal clothing or baggy items of clothing often obscure the true value of the circumference, especially around the waist, leg and arm areas. It is likely that in the case of some subjects that their circumferences were over-estimated due to the amount of clothing that may have been present in some areas of the body during scanning.
    \item \emph{Noise and Sensor Distance}; the amount of noise present in the Kinect meant that our Point Cloud combined with the sensor distance from the subject was of a low resolution and this meant that the gift wrapping algorithm sometimes calculated the circumference around a ring of points selectively or overestimated where there were outliers in the plane.
    
\end{enumerate}

\subsection{Monitoring Limb Changes}

The ongoing monitoring of limb circumference based on this method is considered reasonably accurate for cases where the variation in the circumference is sufficient enough to highlight a difference between it and a previous circumference scan over a period of time. Local changes in the limb such as body fat distribution in particular areas of interest such as the upper arm where the study of the distribution of body fat and skin folds in this area can reveal information about medical conditions or ongoing weight gain/loss from a particular treatment program are harder to identify using this method. This is due to the limitations already outlined with the noise and low resolution of the depth map at local levels of the point cloud unless the limb is of sufficient size.

\pagebreak
\section{Markerless Recognition}
Several tests were conducted, to assess the performance of the proposed solutions to the tracking and registration problems.\\

\subsection{Colour Search}
The testing phase for the colour scanner began with the RGB colour space. Being the simplest to work with, as the imagery is output in ABGR format (simply RGBA in reverse order), the results were obtained very quickly.\\

\subsubsection{Results}
Searching for colours in the RGB colour space was quite effective, but key factors influenced results negatively.\\

Firstly, lighting affects the reflected colour of an object greatly, and is difficult to account for. In uneven or directed lighting most objects will have gradients across them due to the combination of their shape with the direction of the predominant lighting. This means that when searching for the exact colour of an object, the result gives typically very few pixels on the object matching exactly that colour. Here, by ‘exact colour’ we mean the colour of a selected pixel from the object image. \\

This is the key drawback of the RGB colour space for this type of search, which, theoretically, the HSL colour space could help avoid.\\
Allowing for a range of colours, defined by allowing a specified variance above and below the target colour component values, helps to account for the variation in colour and shade on an object.\\

\subsubsection{Choosing Colours}
Dark colours are poor for tracking. One cause in particular is that dark areas are highly susceptible to noise, which is a key burden of working with low-cost hardware. Dark areas rarely contain the target colour in great quantities, rather a miscellany of colour produced by the sensor. Interestingly, this also causes issues when searching for mid-brightness colours; dark regions occasionally cause flecks of brighter colour which may be in the search range.\\

Brighter colours, therefore, are better for this application. Still, a problem remains in deducing which colour is best for tracking in the target environment. Approaching pragmatically, reds are highly present in photography of people, being a strong component in the colour of skin; and is often a primary indicator of skin in image analysis. Red tones are also prominent in wood, thus appearing regularly in many furnishings. Blue is a prominent colour in NHS equipment, clothing, signage and so forth. Green, however, although ubiquitous in natural scenes, is rarely found in office or medical scenes. Certainly in our testing environment, there were very few green objects.\\

To confirm this, we took a pack of brightly coloured papers and attempted to track them. The pack provided paper in green, yellow, orange and pink. For each colour, we tried using a click-to-match function in our micro-application to see if the paper would be tracked at all, and then assess its visibility at 2.2m distance.\\

\begin{table}
\centering
  \begin{tabular}{| r | r | r | r | r |}
    \hline
    Variance & Green & Orange & Pink & Yellow \\ \hline
    0 & 2 & 4 & 5 & 10\\ \hline
    10 & 1500 & 900 & 1100 & 1300\\ \hline
    20 & \textit{2700} & \textit{2000} & \textit{1900} & \textit{2300}\\ \hline
    30 & \textbf{3300} & 2900 & \textbf{2800} & 2700\\ \hline
    40 & 3900 & 4200 & 9000 & \textbf{3600}\\ \hline
    50 & 4900 & \textbf{7000} & 28000 & 4300\\ \hline
    60 & 7900 & 9800 & 47000 & 5000\\ \hline
  \end{tabular}
    \caption{Filter responses to different colour targets with increasing variance. Italics indicate when the shape was fully and evenly shown. Bold indicates the point at which the target center was most accurately found.}
\end{table}


\subsubsection{HSL Colour Space}
The HSL colour space, by separating hue from shade, potentially allows far greater flexibility under different lighting conditions.
Testing the HSL colour space followed the test mantra laid out by the RGB space experiments. Attempting to track the same sheets of paper, it was found that our mechanism in fact made it harder to track. \\

\begin{table}
\centering
  \begin{tabular}{| r | r | r | r | r |}
    \hline
    Variance & Green & Orange & Pink & Yellow \\ \hline
    0 & 0 & 0 & 0 & 0 \\ \hline
    10 & 100 & 500 & 360 & 150 \\ \hline
    20 & \textit{1200} & 1200 & \textit{2300} & \textit{\textbf{1300}} \\ \hline
    30 & \textbf{1900} & \textit{\textbf{1800}} & 4100 & 2400 \\ \hline
    40 & 3900 & 2200 & 7250 & 3900 \\ \hline
    50 & 4500 & 3100 & 13900 & 5871 \\ \hline
    60 & 5300 & 4600 & 59000 & 7700 \\ \hline
  \end{tabular}
    \caption{Filter responses to different colour targets with increasing variance. Italics indicate when the shape was fully and evenly shown. Bold indicates the point at which the target center was most accurately found.}
\end{table}

The results given by the HSL tracker differ greatly in appearance to those from the RGB tracker. Although a similar mechanism is in place, it appears as though the .HSL tracker is more specific; as the variance is increased, the additional pixels shown are not in contiguous clumps as with the RGB tracker. Rather, the new pixels appear sporadically around the image. This could allow better opportunities for noise removal, as the individual pixels can be cancelled out a lot more easily than blocks.\\

\begin{figure}
\centering
  \begin{tabular}{ c }
    \includegraphics[scale=0.25]{zscreenshots/hsl_green_20}\\ 
    \includegraphics[scale=0.25]{zscreenshots/hsl_green_30}\\ 
    \includegraphics[scale=0.25]{zscreenshots/hsl_green_40}\\ 
    \includegraphics[scale=0.25]{zscreenshots/hsl_green_60}\\ 
    \includegraphics[scale=0.25]{zscreenshots/hsl_green_80}\\
  \end{tabular}
    \caption{Increasing the variance begins to allow colours from the environment.}
\end{figure}

Above 2000 found pixels, the added pixels from variance increase were from the environment more than from the paper. As such, the pink paper never gave a useful result for tracking. A major contributor to this was the slight pinkish hue of the paintwork – never particularly noticed until these tests – which came into range as the increased variance allowed for more subtle colours.\\


\subsubsection{Summary}
Despite key obstacles, tracking in the RGB space worked surprisingly well. Based on a simple, almost primitive, idea, it outperformed the, more complex, HSL colour space search quite competitively. Not only that, but it was much faster to compute. The average frame took 300ms with the RGB search, and 700ms for HSL: a significant difference for a system aiming to provide results in real time.\\

\subsubsection{Lighting}
Lighting, as can be expected, did cause issues. The testing area is subject to strong side lighting from the windows, which causes the angle of the paper to affect perceived colour dramatically. This increased the importance of being able to search for a colour range, rather than a specific colour only.\\

An interesting factor that had an effect on the results was the exposure of the imagery. The exposure time seems to quite drastically affect the colours given – especially for the HSL format imagery, which accounts for lightness and saturation – which reduces the effectiveness of the mechanism. Being beyond our control, it adds a layer of complexity to the HSL imagery which would be very hard to account for.\\

Received light at the sensor is not only affected by the light sources, but the reflecting materials in view. The paper in use, being in texture matte rather than shiny, performed better than some other objects tested, which accentuated the specular lighting. There exist materials with ideal optical properties that would reduce such strong lighting effects further, but this begins to run beyond the scope of the project. Nonetheless, the use of a specific material coating on the sensor could be considered an acceptable modification to allow use of the system, and is worth mention.\\


\subsection{Haar}
The generation of the Haar classifier was to be fraught with complex procedures and intricate setup details. It took several weeks’ research and development to train a classifier at all, before testing could begin.\\

Limitations to this progress were numerous. For classifier use, the PCL library imports a classifier from a file, which is generated externally. To train the classifier, a precise setup must be created, and then generated through use of an executable via an obscure command line process. Further, the training process requires large numbers of training images. The documentation suggested minimal numbers in the order of thousands; a quantity which would take quite some time to collect.\\

The first experiment was, therefore, to find the minimum number of samples required to find a simple object. An application was created to allow the import, classification, and export of images into the correct formats and directory structures required for the training process. A sample of fifty positive, classified images and fifty negative images was found to be sufficient to generate a classifier.\\
Initially, a classifier was generated for a spectacles case – a simple object to hand with a featureful logo and interesting texture. The minimal classifier returned no result.\\
 
This test was important, because it raised important issues. First, would it be feasible to spend many more hours collecting sample data?  Secondly, was a Haar classifier a good choice to use at all? The answer, quickly realised, was no. The restrictions in effectiveness and in training and operation mean that for this application Haar classifiers simply aren’t powerful enough. For clear shapes and structures such as faces, they are well suited. But for this application numerous well-trained classifiers would be required to recognise just one object at various rotations. For the intended real-world application, several different sensors may need to be used and tracked, which raises the issue of training many different classifiers with many input images – too large a problem to be tackled.\\


\subsection{SURF}
The SURF classifier was tested on several different types of imagery to assess its performance. With the placement of synthetic features onto the sensor an allowable possibility, tests were devised to test SURF’s performance on both objects and on synthetic features in order to find a setup with sufficient performance.\\

The test suite included:
\begin{itemize}
\item Detection of objects in photographic imagery
\item Feature detection on synthetic features
\item Detection of synthetic features in synthetic imagery
\item Detection of synthetic features in photographic imagery
\end{itemize}
\\
Detection of objects in photographic imagery
Feature detection on synthetic features
Detection of synthetic features in synthetic imagery
Detection of synthetic features in photographic imagery
\\
\subsubsection{Natural Objects}
Images were taken at 2.2m using a digital camera, with a target object in view. Positive samples were taken of the target object at close range. These images were then run through the SURF classifier.
The testing on objects did not provide the positive results expected. Early testing indicated that, even at close range, what appears by eye to be a featureless object indeed has very few features identified.\\

\begin{figure}
\begin{center}
\includegraphics[scale=0.18]{images/20130207_134914.jpg}
\caption{Searching for the target object at distance}
\end{center}
\end{figure}

Part of this is due to SURF’s filtering of features: the initial feature detection will find several thousand features, which are then filtered to retain only those of particular ‘interest’. There is, therefore, no guarantee that the same features will be found in all situations.\\

\begin{figure}
\begin{center}
    \includegraphics[scale=0.18]{images/20130207_134729.jpg}
    \caption{The target object}
\end{center}
\end{figure}

The target close-ups were cropped and cleaned in order to provide a totally featureless background, such that all features found in the images would be on the target. This is a standard practise. Of further note is the use of target imagery at close range, when the target in the field is at distance. The intention was to see whether this aided feature finding, but clearly it does not. For a high-powered system, it might be possible to have either a series of target images at varying distances, or some internal representation of the target that can be used to model it at any distance, but that is not that case.\\

\subsubsection{Synthetic Features}
Various academic papers and examples show SURF working well, but those examples are typically on carefully constrained example objects, and of course don’t show what SURF didn’t work well on. As it has been shown to work poorly on plain objects, the suggestion to devise some synthetic feature to place upon the target seems sensible. \\

\begin{figure}
\begin{center}
    \includegraphics[scale=0.25]{images/synthetic_features_1.png}
    \caption{A synthetic feature inserted into an image, successfully found at near 1:1 scaling.}
\end{center}
\end{figure}

A collection of synthetic features of varying complexity was designed. Each feature was then rescaled in a second image, and the SURF detector used to search for the rescaled feature. The features designed contain different types of features known to be found by the SURF mechanism. \\

The previous problem of unreliable feature sets appeared again at the forefront – when rescaled, the detector found different sets of features. \\

\subsection{The Final Tracking Mechanism}
The system used to track the scanner was the RGB colour search, as it proved under testing to be the most reliable. That said, it was by no means completely reliable. The objects used, particularly under the highly directional lighting of the lab, were subject to specular lighting which reduced the surface area detected. This meant that in around 10\% of frames nothing was found at all. Also with random detected flecks around the room (including edges of plants visible through the glass panel in the door), the feed was quite difficult to work with in its raw form. A very simple mechanism was needed to steady the motion.\\

To perform this the median function was applied on a three frame window. This very effectively covered the holes and anomalies in the data, as neither zeros nor random high or low values would be the median value unless the sensor were detected there for more than one frame - in which case it is possible that the sensor has indeed moved. This made the feed much more stable, but still not completely so.\\

\subsection{Relocating Recorded Scan Positions}
Once stored in the database, recorded scan positions can be re-scanned. The project uses the previously described mechanisms to track the sensor's position relative to the body, except now on every frame. A distance metric is displayed to indicate how close the scanner is to the target position. Again, a capture mechanism is in place to allow the triggering of data capture from a real device.\\

\subsection{Conclusions}
The reliability of the tracking does limit the usability of the system. Overall, what should have been effective display from a set of modern algorithms was hampered by image quality and environmental issues.\\

The low resolution of the imagery, in combination with the environment and the operational properties of the SURF algorithm led to its poor result. As for the Haar classifier, it was quite possibly never going to work. It is difficult to assess how algorithms will perform, without a very full understanding of their operation. Such also are the difficulties of understanding how an algorithm will perform in different applications, scenarios and datasets, that often the best, and only, way to find out is to try.\\

Having made great effort to see a robust, feature-driven algorithm solve the problem, it is disappointing to find that our selections were ineffective, and that it is the almost trivial colour search in the RGB format that provided the only usable solution.\\

\subsection{Further development}
The RGB colour search was rightly the method chosen to drive the sensor tracking. Although lighting affects it so much, it is a fairly stable mechanism under consistent lighting, and is very easy to reconfigure if necessary; much unlike the ordeal required to train a new Haar classifier.\\

\subsubsection{Improving the tracking}
Part of the problem with the RGB tracking mechanism was that the whole environment was searched. One improvement, which was attempted as an aside but not completed, would be to use the depth data as a mask. Removing from the colour frame all regions behind the subjects would reduce drastically the noise present in the image, making tracking simpler whilst also more accurate.\\

The ability to control the lighting itself would be a highly desirable addition.\\



\newpage
\section{Summary}
This section summarises the testing undertaken.\\

\subsubsection{Person Isolation}
The person isolation has been shown to successfully isolate a variety of people and the floor can be isolated at a point cloud level.\\

\subsubsection{Volume Estimation}
An average transform constant has been calculated to be 0.81 (to two decimal places). The height estimation has an error of $\pm$4.25\% within the range tested and can repeatedly measure a subject to within $\pm$2\% on average.\\

The volume estimation could be said to have a minimum accuracy of between $\pm$1.06\% and $\pm$9.12\%. Whereas in practice the average error is 23.05\%, although this may be down to clothing and the relative ratios of subject volume is preserved to 0.62\%, but more test subjects are needed before any real conclusions can be made.\\

\subsubsection{Markerless Recognition}
Even with motion smoothing, the result from the RGB-based tracking mechanism is very shaky, and is strongly affected by lighting and the environment. Tracking is possible, but is not very reliable. The translation to real-world coordinates is largely accurate when the target tracking is stable, but only to +/- 2mm due to the resolution of the imagery.\\