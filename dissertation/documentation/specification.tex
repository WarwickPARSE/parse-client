\chapter{Introduction}
\label{specification}
\setcounter{page}{1}
\pagenumbering{arabic}

\section{The PARSE Project}
The overall project deliverable was to create a toolkit to measure and monitor body volumes, as well as limb circumferences and relative positions of ultrasound scanners on the surface of the body.\\ 

There were three key aims for the toolkit:\begin{itemize}
\item To offer a similar accuracy of volume measurement when compared to the BOD POD \cite{cosmed2013}, at a much lower cost 
\item To be able repeatedly measure limb circumferences 
\item To repeatedly position an ultrasound scanner without the use of an indelible mark\end{itemize}

The team was also required to provide an interface that would allow medical personnel to use the toolkit with ease.

\section{Problem Statement}
\label{spec: problem statement}
Computer Vision and object recognition are becoming more prevalent in day-to-day life, entering the home through commodity hardware such as the Microsoft Kinect. Because of this, many fields are looking to use the Kinect to automate or improve many of their processes. One such field is medicine, where medical imaging, specifically range imaging is rapidly developing in a number of different contexts, such as patient setup, and being applied in a variety of situations. Advancements in this area are leading to better patient monitoring, diagnosis and treatment of patients in various sub domains in Medicine.\\  

The PARSE project aims to create an application toolkit to increase the reliability and accuracy of the measurements taken when monitoring weight loss and body size variations of a patient. The toolkit will provide an easy to use interface allowing for medical personnel to take measurements of body volume, limb circumferences and relative positions of ultrasound scanners on the surface of the body. These measurements can be stored persistently and be recalled later as part of a patient’s medical record. The remit for achieving this will be in our use of commodity hardware; specifically, the Microsoft Kinect.\\ 

\section{Motivation}
\label{spec:motivation}
The motivation for this project came from several limitations associated with cost and the accessibility of current practice for measuring weight loss, body shape and the calibration/configuration of non-invasive medical scanning equipment. Body volume is typically measured using multiple sensors \cite{Bauer2011} or expensive air displacement plethysmography equipment \cite{Izadi2011}. A need was identified for a system that could accurately measure a patient's total body volume using a single piece of commodity hardware and means of image processing.\\ 

The traditional method of measuring circumference of body parts such as arms or torso is limited by the accuracy of the physical device used to measure it, usually a spring tape. A bigger problem, however, arises when taking multiple readings. For the measurements to be useful the exact same point must be measured each time. A need was stated for the new toolkit to accurately identify the circumference of individual body parts, either from the original body scan or a new individual scan of the limb, and to enable the registration of and consistent guidance to the point of measurement.\\ 

The depth of subcutaneous fat in a person can also be measured to determine a patient's weight loss or variation in body size, and the depth of the fat is also an indicator of other serious conditions such as insulin resistance \cite{Goodpaster1997} and coronary heart disease \cite{Ducimetiere1986}. The problem here is measuring the depth of this subcutaneous fat consistently by placing the ultrasound scanner in precisely the same place, at the same angle, when taking multiple readings of the same body part over time. If the positioning is not identical, errors can be introduced and may give a false indicator of loss or gain in subcutaneous fat. The PARSE team feel that the proposed toolkit can aid this, by recording the position of the first scan and providing direction to the user for subsequent scans. This guided positioning will be achieved using a mixture of the Kinect skeletal tracking and object recognition algorithms on the depth and image feeds from the Kinect scanner.\\

\section{Project Customer}
The project team first approached the customer at the start of the project with a proposal for developing an innovative object sharing environment. Matt continued to support the PARSE project as the project evolved away from this initial idea and throughout the entire duration.\\ 

A former postgraduate Computer Science Research Assistant at the University of Warwick, Matt was considered appropriate for the role of project customer due to his experience authoring many academic publications, all of which required some degree of collaboration, which makes him familiar with the challenges and complications of working in teams and group work. He was also approached due to his friendly and easy going attitude.\\

\section{Project Deliverables}
The project deliverables were defined by the CS407 \cite{ouroboros2007} course structure. Therefore the deliverables were: \begin{itemize}

\item The \textb{Project Specification}: The specification was not marked as a separate component, but was a compulsory milestone that was required to be passed for the project to continue. It identified the planned activities and schedule for the remainder of the project. The document has been included in Appendix A.

\item \textb{Progress Presentation}:
The progress presentation took the form of a poster presentation during week 10, and counted for 10\% of the assessment. The poster is included in Appendix B and the accompanying slides are also included in Appendix B.

\item The \textb{PARSE Toolkit}: The substantial software outcome, specified by the Project Specification and described in full in this final report. The software itself carries no marks.

\item \textb{This Report}: Is jointly authored by the members of PARSE and counts for 60\% of the assessment. It's purpose is to document the process of delivering the agreed software, from both a technical and a process management standpoint. 

\item The \textb{Individual Report}s: Pieces of reflective writing which allows each group member to discuss their own contribution to the project, and what lessons they have learned from the process. It counts for 10\% of the assessment.

\item The \texb{Final Project Presentation}: Will count for the remaining 20\% of the assessment, allows the group to present their completed work to an audience, and allows members of the audience to question members of the group about their work. The group will be expected to demonstrate their software working.\end{itemize}

\pagebreak

\section{Requirements}
\label{spec:requirements}
This section enumerates the functional and non-functional requirements of the projects and presents the rationale behind them.\\


\subsection{Functional}
\label{spec:functional}

Our functional requirements have been maintained throughout the project and subject to minor alterations, have been satisfied to a reasonable extent. The meeting of these requirements is discussed in the evaluation section of the report.

\begin{longtable}{|p{0.5cm}|p{5cm}|p{6cm}|} \hline
   & Requirement & Rationale \\ \hline
  1 & The system should be able to retrieve a point cloud representation from the Kinect sensor by mapping the colour stream onto points of a depth map. &
    We wish to estimate volume based on the data from the depth and RGB colour camera. A coarse point cloud will produce a reasonable approximation of the volume. \\ \hline
      2 & The system should be capable of identifying both large and small objects within a relatively complex scene which may be subject to noise or partial occlusion and extract that from the scene. &
    To estimate the volume of the human body or be able to register the presence of an object in the scene used for patient scanning, we need to effectively segment the scene into it's constituent parts. \\ \hline
      3 & The system should be capable of storing point clouds in a data structure that allows efficient access. &
    Point clouds will be subject to operations that are computationally complex for the purposes of rendering and manipulation. A suitable data structure will need to be selected that supports efficient operation. \\ \hline
      4 & The system should be capable of dealing with both partial and complete body scans. &
    A complete body scan will be required to generate an overall approximation of body volume subject to environmental interference. Partial scans will be requied in the context of scanning where full volume estimation isn't required, such as in the instance of point registration on the patient. \\ \hline
      5 & The system should be capable of recognising, storing and recalling the location and orientation of a known mobile object or scanning device relative to a static object. &
    The scanning of various areas of the body for the monitoring of size and body variation over time (such as periodic ultrasound scanning) relies on the scan being taken from the same location with a similar orientation to the previous scan, hence requiring accurate location and positioning for repeated measurements. \\ \hline
      6 & The system should be capable of directing the user to place the sensor at the previous measurement point on the body and detecting when the sensor has been positioned and oriented correctly. &
    To obtain measurements useful for comparison, the measurement should be taken at precisely the same point on the body, with the sensor at the same orientation. \\ \hline
      7 & The system should be capable of measuring circumferences and local volumes for particular areas of the  body. &
    The measurements of limb circumferences will be useful for the monitoring of local changes in size or depth on the surface of the body. \\ \hline
    
\end{longtable}

\pagebreak 

\subsection{Non Functional}
\label{spec:non functional}

The non functional criteria for the project have been tested and evaluated later in the Evaluation and testing section of the report.

\begin{longtable}{ | p{0.5cm}|p{5cm}|p{6cm}|} \hline
   & Requirement & Rationale \\ \hline 
  1 & The system should provide an intuitive and easy to use mechanism for scanning. &
    The scanning method should have a limited training overhead so that it can be used effectively and widely adopted amongst the user base. \\ \hline
      2 & The system should be efficient, aiming to return rendered results in real time. &
    Computation performed on the point cloud is likely to be expensive but results from processing should be returned as quickly as possible for usability purposes. \\ \hline
      3 & The system should be able to estimate measurements accurately. &
    Body volume and circumferences of the body need to be reasonably estimated in order to provide a clear indicator for future scans. \\ \hline
      4 & The system should be capable of being installed with minimal pre-requisite software/hardware. &
    The system will likely be deployed in a number of medical contexts and apart from the requirement of a Microsoft Kinect and a capable graphics card, pre-requisite software will be bundled as part of the system. \\ \hline
    
\end{longtable}
