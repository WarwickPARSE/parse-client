\label{abstract}
\begin{abstract}

Measuring and monitoring the composition of the body with respect to it's overall size, circumferences and the levels of subcutaneous fat in particular areas of the body is important for the diagnosis, treatment and prevention of diseases and weight related conditions. The capturing of key metrics such as body volume and limb circumference has been demonstrated to be important to the prognosis of patients who are undergoing weight changes or related treatment programmes. The use of displacement and conventional circumference measurement has provided reasonable accuracies (+/-5\%) but are subject to errors over time which are difficult to track or monitor. More recently, conventional image processing techniques are being used to capture these particular metrics. The PARSE project has looked into the use of the Microsoft Kinect as a means of capturing and monitoring overall body volume and limb circumference in a novel configuration which requires significantly less setup and operational complexity compared to similar body scanning solutions. In testing accuracies within 5\% on initial scans and around 10\% over repeated scans were recorded with future optimisations possible given the active nature of the Kinect Development Environment.\\

\emph{\bf{Keywords: }} Kinect, volume, circumference, registration, markerless

\end{abstract}