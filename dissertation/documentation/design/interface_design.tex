\section{Interface Design}
\label{design:interface}

The interface was designed using the WPF framework. Each component of the user interface was written to support it's respective functionality and each UI unit is self contained. The use of a centralised class for dispatching events and handling core functionality means that data present in each functional UI unit of the system can be accessed via the relevant procedures in the CoreLoader. This allows the respective functions access to each others data and methods where appropriate.

\subsection{Intended Audience}

It is anticipated that this project will be used by researchers or medical practitioners on a initial patient examination level rather than on a diagnostic basis. Therefore the functionality of the system and the results it returns need to presented in a way which can be easily understood by both medical practitioners and the general public. This is as per the non-functional requirements facilitated through an adequate user interface.

\subsection{Component Design}

For the design of each component of the UI, the group agreed to follow the MVVM design pattern where the business and presentation logic of the system is separated from the technicalities of implementing the user interface. This meant that we could use the user interface to expose the data objects generated by visualisations or calculation of volumes or circumferences. WPF does not actually support the MVVM pattern of design due to all user interface components using an underlying XAML representation with access to data objects permitted by a data binding at run time. This required us designing the interface that would emulate such functionality. This was achieved by using separate user controls that would bind to each data object generated by the functionalities of the system.

Each user interface component uses a \texttt{\_Loader} convention to signify it's relation to the CoreLoader interface. The CoreLoader UI component is the parent class of all other UI components. This means child components can consist of other windows or custom user controls. It also reflects the architecture of the system. 

\subsubsection{CoreLoader}

This component is the owner of all other user interface components. This means that any global operations applied to this component affect all other currently open components. CoreLoader provides access to all the functions of the system.

\subsubsection{ScanLoader}

ScanLoader contains the viewer for visualising models representative of scanned pointclouds. ScanLoader allows for interaction with the model including panning and zoom as well as permitting aesthetic changes to the model such as colour or rendering.

\subsubsection{ViewLoader}

This component shows the raw data and overlay streams of the Kinect device. ViewLoader is capable of representing colour, depth and skeletal as well as feeds that incorporate additional post processing such as depth isolation or color isolation of a given subject.

\subsubsection{HistoryLoader}

HistoryLoader shows the results of any volume or limb circumference calculations carried out on a patient. These calculations will also be visualised depending on if planes need to be visualised for volumes. HistoryLoader also contains charting of results according to previous scans that the patient may have been subjected to in order to show changes in measurement for both limb circumference and volume.

\subsubsection{PatientLoader}

PatientLoader is a component that allows for the entry, editing and viewing of details related to the patient that is stored on the system database. 

