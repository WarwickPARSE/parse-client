\section{Limb Circumference}
\label{design:limbcircum}

The design of algorithms used for calculating limb circumference builds on the design principles of planimetry and as identified in section 3.6.1, this forms a reasonable basis for calculating circumferences using range imaging. 

\subsection{Plane Partitioning}

The limb circumference approach uses the sames SSP methodology for the estimation of volumes, pulling out planes using the shoelace formula. Hence limb circumference will operate on the same basis of the full body scan. Due consideration was given to alternative scanning contexts, where localised scanning would take place around a particular limb. However, as per the functional requirements of the system, we wanted to minimise the amount of configuration and patient setup required for the capturing of these measurements. The resolution of the Kinect device for capturing and tracking limb circumference, with it's 11-bit depth and 2048 levels of sensitivity was deemed adequate for estimating these metrics at distance. \\

Plane partitioning is achieved by using a mapping between the skeletal feature points and the point cloud isolated from the depth map. The 20 skeletal feature points are defined in terms of \emph{real-world coordinates} inferred from the depth map representation. The $(x,y,z)$ co-ordinates of these feature points are related by the inferred body part identification. With this, a bounding box around the limb is first determined in real-world coordinates. These bounds are defined by taking the minimal and maximal x,y coordinate's from the feature points for the limb. These bounds are then transformed into the point cloud co-ordinate system. The relative depth of the limb skeletal feature points are identical to the depth of the point cloud and as such do not require this transformation. \\

\begin{figure}[ht]
\begin{center}
$PC_n_e_w(x,y) = ((C_x,C_y) - (x,y)) * zz * f_x_,_yinv$
\end{center}
\caption{Translation from Real-world to Point Cloud co-ordinate system}
\label{fig:conversionformula}
\end{figure}

$C_x$ and $C_y$ are defined as the centre of the captured depth map of resolution 640x480 pixels. The depth map is then scaled by a constant $zz$ defining the size of the points that will be used to represent the point cloud in the 3D visualisation. The point is then multiplied by the invariants $f_x_,_yinv$. The importance of using this arbitrary point cloud space is highlighted by the unique set of points it generates that can then be represented accurately. It also allows information defined in real-world coordinates such as a the aforementioned skeletal data to be associated with each captured point cloud.

\subsection{Circumference Calculation}

The algorithm then subsamples on the n returned planes from the partitioned point cloud but uses the planes equidistant between the limb joints to give a mean representation of limb circumference on the selected limb. The circumference of each plane is calculated using the \emph{Gift Wrapping} algorithm, as detailed in the research section of the report, which returns the convex hull of a set of points. In the case of the PARSE implementation, we pass a subsampled region of planes and for each y return a convex hull which is then averaged for the mid-region for the particular limb. Similar to the estimation of Volume, the circumference returned is not defined in terms of SI units but rather determined in terms of the point cloud. This will require further testing to ascertain a suitable multiplicative constant to obtain a reasonable approximation to the circumference of the subject. \\

\subsection{Algorithm Running Time}

This algorithm enumerates each limb and performs plane partitioning and circumference calculation in real time. Because the algorithm itself is derived from the volume calculation and planimetry used to calculate whole-body volume, it is sensible to assume a similar running complexity to $O(\sqrt[3]{p}^2)$. The running complexity of calculating limb circumference is further compounded by the limb enumeration and bounding box calculation step where $O(n)$ represents the possible number of times the planimetry algorithm is to be run and $O(n^2)$ for each bound for those limbs to be calculated.

\subsection{Alternative Approaches}

Apart from the direct partitioning of this abstract point cloud structure and the calculation of the convex hull for each of the extracted planes, there are other means of limb circumference measurement that uses least squares regression for the fitting of ellipses to scatted 3 dimensional data, something that could be easily extended to calculation of circumference for a given point cloud region based on the fitted ellipses. Fitzgibbon et al. \cite{Fitzgibbon1999} fit these primitive ellipses to provided image data by finding the set of parameters that minimise the distance between the provided data points and the proposed ellipse shape.  While this least squares method is only used on scattered data in 2 dimensional space, this method could be viably extended into 3D space as has been achieved by Geiger who proposes a method which registers the persons point cloud in different poses and projects the captured 3 dimensional data into 2D space for registering the circumference as the person rotates \cite{Geiger2011}.


